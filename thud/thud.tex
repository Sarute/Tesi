
%% Le lingue utilizzate, che verranno passate come opzioni al pacchetto babel. La prima indicata (a differenza del solito) sar‡ quella primaria.
%% Se si utilizzano una o pi˘ lingue diverse da "italian" o "english", leggere le istruzioni in fondo.
\def\babelopt{english}
%% Valori ammessi per target: bach (tesi triennale), mst (tesi magistrale), phd (tesi di dottorato).
\documentclass[target=mst,babel=\babelopt]{thud}[2014/01/17]

%% --- Informazioni sulla tesi ---
%% Per tutti i tipi di tesi
\title{Retrospettiva antero-posteriore dei linguaggi funzionali imperativi}
\author{Sara Fabro}
\supervisor{Dott.\ Marino Miculan}
%% Altri campi disponibili: \reviewer, \tutor, \chair, \date (anno accademico, calcolato in automatico).
%% Con \supervisor, \cosupervisor, \reviewer e \tutor si possono indicare pi˘ nomi separati da \and.
%% Per le sole tesi di dottorato
\phdnumber{313}
\cycle{XXVIII}
\contacts{Via della Sintassi Astratta, 0/1\\65536 Gigatera --- Italia\\+39 0123 456789\\\texttt{http://www.example.com}\\\texttt{inbox@example.com}}
\rights{Tutti i diritti riservati a me stesso e basta.}

\usepackage{hyperref}
\usepackage{url}
%% Campi obbligatori: \title e \author.

%% --- Pacchetti consigliati ---
%% hyperref: Regola le impostazioni della creazione del PDF... pi˘ tante altre cose.
%% tocbibind: Inserisce nell'indice anche la lista delle figure, la bibliografia, ecc.

%% --- Stili di pagina disponibili (comando \pagestyle) ---
%% sfbig (predefinito): Apertura delle parti e dei capitoli col numero grande; titoli delle parti e dei capitoli e intestazioni di pagina in sans serif.
%% big: Come "sfbig", solo serif.
%% plain: Apertura delle parti e dei capitoli tradizionali di LaTeX; intestazioni di pagina come "big".

\begin{document}


%% Il frontespizio prima di tutto!
\maketitle

%% Dedica (opzionale)
\begin{dedication}Al mio cane,\par per avermi ascoltato mentre
  ripassavo le lezioni.\end{dedication}

%% Ringraziamenti (opzionali)
\acknowledgements 

%% Sommario (opzionale)
\abstract

%% Indice
\tableofcontents

%% Lista delle tabelle (se presenti)
% \listoftables

%% Lista delle figure (se presenti)
% \listoffigures

%% Corpo principale del documento
\mainmatter

%% Parte La suddivisione in parti Ë opzionale; talvolta sono
%% sufficienti i capitoli.
\part{Parte}

%% Capitolo
\chapter{Introduction}
Developing reliable software is becoming increasingly
difficult. Software's insecurity is the result of the complexity of the modern software system, the number of
people involved in developing, the lack of time due to the market
request, and others factors. Computer Science and Software Engineering
are working at different layers with the aim to improve software's
reliability: from managing software projects and
structuring programming teams to programming languages and patterns to
matheamtical techniques for specifying and certifying software properties.


%% Sezione
\section{Certified Software}

Even if we don't think about it, software is involved in each aspect
of our lives. Communication and computing devices, consumer products (such as cameras,
refrigerators, thermostats), aviation systems, medical devices; all of them are controlled by software. 

Despite this, software is still developed without a precise
specification. The outcome of heedful software engineering phase is
not always taken in consideration by developers, which often start to
write code with an unclear or informal specification. As a result, a developer may
deliver a program that will not fulfill the specification.
Wide testing and debugging may contract the gap between the program
delivered and the original program project, but there is no assurance
that the program will satisfy the entire specification. Such inaccuracy may
not always be relevant, but when joining different modules together,
these approximations may lead to a system that nobody can manage or,
even worse, understand. It is diffusely accepted that bugs in software
are a grave problem today, since they increase the cost of software
development and decrease the security of the system.

A developer that write code without a formal specification, can be compared to the
mathematician describe by Bourbak, in his book: \cite{Bourbaki98}
\begin{quote}
"his experience and mathematical flair tell him that translation into formal language would be no 
more than an exercise of patience (though doubtless a very tedious one)''
\end{quote}
Without a formal specification, one can occur in mistakes due to a
wrong line of reasoning, like an inappropriate use of deduction or an
inappropriate analogy between two different cases. The research
community has developed a large number of techniques for finding bugs
in programs or even proving programs to be free of certain classes of
bug. The drawback of these approaches is that they are used after the
programming task, while the idea is to apply formal methods throughout
the software lifecycle.

Certified software consists of an executable program C plus a formal
proof P that the software is free of bugs, with respect to a particular
dependability claim S. \cite{Shao10}. The outcome
software will certify the behavior described by the claim, which
can range from making almost no guarantee to simple type safety property, or all
the way to deep security and correctness properties. Because the claim
comes with a mechanized proof, the dependability can be checked
independently and automatically in an extremely reliable way.

The conventional wisdom is that Certified Software will never be
practiced because of two reasons: first, any real software must also rely on the underlying runtime system
which is too low-level and complex to be verifiable; then, as a second
reason, a significant part of developers asserts that the cost of a
formal verification is too high compared with his benefits.
In the 21st century, however, the progress and the interests in practical application of
interactive theorem proving increased significantly. In the realm of
pure mathematics, Georges Gonthier built a machine-checked proof of 
the four-color theorem \cite{Gonthier08}. His achievement is the result of six years of
work, 170.000 lines of code, 15.000 definitions and 4.300 theorems. 
In the realm of proving verification, Xavier Leroy developed the
CompCert, a verified C compiler back-end\cite{Compcert}.
Both these two projects and many others \cite{JSCert} are developed and
checked using the Coq Proof Assistant.


Coq is a general purpose environment for developing mathematical
proofs and not a dedicated software verification tool.
However, it is based on a powerful language including basic functional
programming and high-level specifications.
As such it offers modern ways to literally program proofs in a
structured way with advanced data-types, proofs by computation, and general purpose libraries of definitions and lemmas.
Coq is well suited for software verification of programs involving
advanced specifications (language semantics, real numbers). 
\cite{paulinintroduction}


\section{Limits of Coq Extraction}
As mentioned before Coq isn't a dedicated software verification tool,
but it offers an extraction modul for automatic generation of
programs out of Coq proofs and functions. The extraction process
 is based on the Curry-Howard isomorphism, which asserts that a constructive proof is isomorphic to
a functional program.  The main motivation for this feature is to produce certified software: each
property proved in Coq will still be valid after extraction. Certified
code obtained this way can be easily integrated in larger
developments, so that wider communities of programmers can benefit from extraction.
The languages supported by the module are three:
OCaml, Haskell and Scheme.


Extraction isn't a new
and revolutionary idea since it exists in Coq since 1989 and it is
present in other theorem provers, like Nuprl\cite{Nuprl}. Unfortunately,
the extraction process has still one significant limit: it's all
about pure functional languages. 
Non-functional programming languages hardly feature a type
theory supporting a Curry-Howard isomorphism, and even if such a
theory were available, implementing a proof-assistant with its own
extraction facilities wouldn't be a valid solution.
Non-functional programming languages such as C or Java do include
computational effects, for example a Java function may throw an
exception or modify a state during a computation.

 What should a developer do in order to have certified software, with
 side effects, extracted directly from a Proof Assistant?

\section{Thesis proposal}
The thesis aims to extract certified Haskell code with side-effects
from the Coq Proof Assistant. In order to achieve this goal, it is 
needed to formalize non-functional aspects. A solution to this problem
is represented by the use of monads.
In Computer Science, a monad describes a "notion of computation''
\cite{Moggi91}. It
is a map that sends every type X of some given programming language to
a new Type T(X), that is the type of T-computations with values in X. 
The proposal is to extend the Coq proof assistant with a
suitable computation monad, that will cover the specific
non-functional computational aspect. This solution is inspired by
others pure functional languages, such as Haskell.

METHODOLOGY

\section{Other solutions to the problem}
Formal verification of side effect aspects is a hot topic in the
Computer Science community. There are several teams and projects that
are working on it, with different point of view and approach.
Cybele\cite{Cybele} is a tool to write proofs by reflection in Coq;
it is mainly developed at the Inria and the Université Paris Diderot-Paris 7.
Code is written in the Coq Proof Assistant extended with effects and
termination aspects,s then procedures are extracted to OCaml. They
present an innovative technique for proof-by-reflection. The key
concept is the use of simulable monads, where a monad
is simulable if, for all terminating reduction sequences in its equivalent
effectual computational model, there exists a witness from which the same
reduction may be simulated a posteriori by the
monad.\cite{claret:hal-00870110}
Another project is the one conducted by Jean-Guillaume Dumas, 
Dominique Duval, Burak Ekici, Damien Pous. They work on Formal
verification in Coq of program properties involving the global state
effect. \cite{Dumas:2014:coqstates} Their Coq framework allows to
verify properties about the manipulation of the global state
effect. The state doesn't appear clearly with its type of expressions
which manipulate it, but it's a combination of decorations added to terms and
equations. As a first result they give the proof of commutation
update-update,  but their main goal is to generalize to the other
side effects in order to reach the verification of real-life C code with effects.
The point of view changes with the approach adopted by Gabe
Dijkstra.\cite{Dijkstra12} His idea is to automate the process of
translating Haskell code into the Coq Proof Assistant, in such way
that the extracted Haskell code from Coq will have the same interface
and semantics.  


\chapter{Haskell}

Having a list of statements, a list of programming languages and
ranking those languages in order of how well the statements apply to
them, pointed out Haskell as the number one for both the following
statements:
"Learning this language significantly changed how I use other languages''
\cite{interview1}
"I would recommend most programmers learn this language, regardless of
whether they have a specific need for it."
\cite{interview2}
This result isn't an unexpected surprise, since Haskell it's quite
different from most other programming languages. 

This chapter will give an overview of Haskell,
since it is the target language of the methodology
presented in this thesis. Particular attention will be given to monads
and how they are embedded in the programming language.


\section{What is Haskell?}
Haskell is a static, lazy, pure functional language,
 based on the lambda calculus (hence the lambda in the
logo), and designed over a period of three years by a group of
Computer Scientists from the functional programming community.
It is named after Haskell Brook Curry, an American mathematician and
logician. The design effort came about because of the perceived need
for a new common functional language with these constraints:
\begin{enumerate}
  \item It should be suitable for teaching, research, and
    applications, including building large systems.
    \item It should be completely described via the publication of a
      formal syntax and semantics.
      \item It should be freely available. Anyone should be permitted
        to implement the language and distribute it to whomever they
        please.
        \item It should be based on ideas that enjoy a wide consensus.
          \item It should reduce unnecessary diversity in functional programming languages.
\end{enumerate} \cite{haskell98}


Haskell is lazy, it won't execute or
calculate a function until it's forced to determine a
result. This characteristic is suitable with referential transparency,
which guarantees that a function with the same parameters always evaluates the same result
in any context. The two features combined together allows to think of
Haskell's programs as a series of transformations on data.


Java, C, Pascal, and many others are all imperative languages. They
consists of a sequence of commands, executed with a precise
and not mutable order. Haskell supports functional programming
instead. The main program itself is written as a function which receives the
program’s input as its argument and delivers the program’s output as its result.
Typically the main function is defined in terms of other functions, which in
turn are defined in terms of still more functions, until at the bottom level the
functions are language primitives.\cite{hughes:matters-cj} Haskell
isn't simply one of many functional programming languages, it's a pure
functional programming language: it does not allow any side-effect.

Haskell is a pure language; the evaluation of a
program is equivalent to evaluating a function in the pure
mathematical sense. Purity leads up pervasive consequences, since side
effects are undoubtedly very useful. The lack of side effects was
heavily perceived at the beginning, Haskell I/O was a complete and
utter confusion. Necessity being the mother of invention, this
shortcoming led to the invention of monadic I/O, considered one of the
Haskell's main contribution to the programming world.

Haskell is statically typed, its programs are compiled before run.
Its compiler will catch contingent errors at compile time, 
rather than finding them during the production stage. 
Additionally, Haskell has a good type system with type inference.
As a consequence, there is no need to declare each argument's type of
functions, since the type system can intelligently figure it out from
its own. Type inference also allows your code to be more general. 
If a function takes two parameters and adds them together and 
it isn't explicitly stated their type, the function will work on any two parameters that act like numbers.



\section{What can Haskell offer to the programmer?}

\subsection{Purity}
As mentioned before, unlike some other functional programming
languages, Haskell is pure. The result of a function is determined by
its input, and only by its input; no side effects are
expected. Haskell developers think
about what the program is to compute not how or when it's computed,
interesting even more when programming parallelize threads.
On the contrary, the close relationship between imperative
languages and the execution from the processor of sequencing commands
implies that imperative languages can never rise above the task of
sequencing.
Purity it's also important because it prevents mistakes due to side
effects and combined with polymorphism, it encourage a style of code
that is modular, refactorable and testable.


\subsection{Modularity}
Modularity is a key concept to successful programming. When trying to
write a program, one first splits the problem into sub-problems, then
solves the bottom problem and tries to combine the solutions
together. The ways in which one can divide the original problem is
heavily influenced by the ways in which he can "glue'' solutions
together.
A well-know analogy is the construction of a wooden chair. If the part
are made separately and then glued together, the task is solved in a
easy way. But if the chair has to be carved out of a solid piece of
wood, it would become obviously harder. John Hughes used this
comparison in his paper:

\begin{quote}
"Languages which aim to improve productivity must support modular
programming well. But new scope rules and mechanisms for separate
compilation are not enough - modularity means more than modules.
 Our ability to decompose a problem into parts depends directly on our
 ability to glue solutions together.
 To assist modular programming, a language must provide good glue.

Functional programming languages provide two new kinds of glue -
higher-order functions and lazy evaluation."
\cite{hughes:matters-cj}
\end{quote}

\subsection{Elegant code}
Haskell is often described as a "beautiful'', "elegant" or even "cool"
language, a description that is hardly associated  with the committee
designs for a new language. Despite that, there are some factors that
have contributed to give Haskell this reputation.
First of all Haskell was needed at that precise time and the goals
among the committees were aligned. To understand the favourable
situation in which it began to be developed, it's sufficient to cite
some sentences of the Turing Award lecture delivered by John Backus in
1978:

\begin{quote}
"Can Programming Be Liberated from the von 
Neumann Style? A Functional Style and Its 
Algebra of Programs


... An alternative functional style of programming is 
founded on the use of combining forms for creating 
programs. Functional programs deal with structured 
data, are often nonrepetitive and nonrecursive, are hierarchically constructed, do not name their arguments, and 
do not require the complex machinery of procedure 
declarations to become generally applicable. Combining 
forms can use high level programs to build still higher 
level ones in a style not possible in conventional languages.

...Associated with the functional style of programming 
is an algebra of programs whose variables range over 
programs and whose operations are combining forms. 
This algebra can be used to transform programs and to 
solve equations whose "unknowns" are programs in much 
the same way one transforms equations in high school 
algebra. These transformations are given by algebraic 
laws and are carried out in the same language in which 
programs are written. Combining forms are chosen not 
only for their programming power but also for the power 
of their associated algebraic laws. General theorems of 
the algebra give the detailed behavior and termination 
conditions for large classes of programs." \cite{Backus:1978:CPL}

\end{quote}

One of the committee's priorities was the mathematical notation: clear, intuitive
and elegant; to the detriment of a formally defined semantics. Many
debates were stress by cries of "Does it have a compositional
semantics?'' or "What does the domain look like?''.
But in the end the absence of a formal language definition allows the
language to evolve easily, because the costs of
producing fully formal specifications of any proposed change are
heavy, and by themselves discourage changes.

\subsection{Monads}
As stated before, the benefits of pure functions are that they are
well-behaved (given a particular input, they will always compute the
exact same output), easy to test, completely predictable, and less to
prone to bugs. But programming with only pure functions is
limiting, indeed without side effects writing a simple program that
copy a file form one directory to another will be an impossible task.

%% Fine dei capitoli normali, inizio dei capitoli-appendice
%% (opzionali)
\appendix

\part{Appendici}



%% Parte conclusiva del documento; tipicamente per riassunto,
%% bibliografia e/o indice analitico.
\backmatter

%% Riassunto (opzionale)
\summary Maecenas tempor elit sed arcu commodo, dapibus sagittis leo
egestas. Praesent at ultrices urna. Integer et nibh in augue mollis
facilisis sit amet eget magna. Fusce at porttitor sapien. Phasellus
imperdiet, felis et molestie vulputate, mauris sapien tincidunt justo,
in lacinia velit nisi nec ipsum. Duis elementum pharetra lorem, ut
pellentesque nulla congue et. Sed eu venenatis tellus, pharetra cursus
felis. Sed et luctus nunc. Aenean commodo, neque a aliquam bibendum,
mauris augue fringilla justo, et scelerisque odio mi sit amet
diam. Nulla at placerat nibh, nec rutrum urna. Donec ut egestas
magna. Aliquam erat volutpat. Phasellus vestibulum justo sed purus
mattis, vitae lacinia magna viverra. Nulla rutrum diam dui, vel semper
mi mattis ac. Vestibulum ante ipsum primis in faucibus orci luctus et
ultrices posuere cubilia Curae; Donec id vestibulum lectus, eget
tristique est.

%% Bibliografia (opzionale)
\bibliographystyle{plain_\languagename}%% Carica l'omonimo file .bst, dove \languagename Ë la lingua attiva.
%% Nel caso in cui si usi un file .bib (consigliato)
\bibliography{thud}
%% Nel caso di bibliografia manuale, usare l'environment
%% thebibliography.

%% Per l'indice analitico, usare il pacchetto makeidx (o analogo).

\end{document}

--- Istruzioni per l'aggiunta di nuove lingue ---
Per ogni nuova lingua utilizzata aggiungere nel preambolo il seguente spezzone:
    \addto\captionsitalian{%
        \def\abstractname{Sommario}%
        \def\acknowledgementsname{Ringraziamenti}%
        \def\authorcontactsname{Contatti dell'autore}%
        \def\candidatename{Candidato}%
        \def\chairname{Direttore}%
        \def\conclusionsname{Conclusioni}%
        \def\cosupervisorname{Co-relatore}%
        \def\cosupervisorsname{Co-relatori}%
        \def\cyclename{Ciclo}%
        \def\datename{Anno accademico}%
        \def\indexname{Indice analitico}%
        \def\institutecontactsname{Contatti dell'Istituto}%
        \def\introductionname{Introduzione}%
        \def\prefacename{Prefazione}%
        \def\reviewername{Controrelatore}%
        \def\reviewersname{Controrelatori}%
        %% Anno accademico
        \def\shortdatename{A.A.}%
        \def\summaryname{Riassunto}%
        \def\supervisorname{Relatore}%
        \def\supervisorsname{Relatori}%
        \def\thesisname{Tesi di \expandafter\ifcase\csname thud@target\endcsname Laurea\or Laurea Magistrale\or Dottorato\fi}%
        \def\tutorname{Tutor aziendale%
        \def\tutorsname{Tutor aziendali}%
    }
sostituendo a "italian" (nella 1a riga) il nome della lingua e traducendo le varie voci.
