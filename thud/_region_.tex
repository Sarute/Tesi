\message{ !name(thud.tex)}
%% Le lingue utilizzate, che verranno passate come opzioni al pacchetto babel. La prima indicata (a differenza del solito) sar‡ quella primaria.
%% Se si utilizzano una o pi˘ lingue diverse da "italian" o "english", leggere le istruzioni in fondo.
\def\babelopt{english}
%% Valori ammessi per target: bach (tesi triennale), mst (tesi magistrale), phd (tesi di dottorato).
\documentclass[target=mst,babel=\babelopt]{thud}[2014/01/17]

%% --- Informazioni sulla tesi ---
%% Per tutti i tipi di tesi
\title{Retrospettiva antero-posteriore dei linguaggi funzionali imperativi}
\author{Sara Fabro}
\supervisor{Dott.\ Marino Miculan}
%% Altri campi disponibili: \reviewer, \tutor, \chair, \date (anno accademico, calcolato in automatico).
%% Con \supervisor, \cosupervisor, \reviewer e \tutor si possono indicare pi˘ nomi separati da \and.
%% Per le sole tesi di dottorato
\phdnumber{313}
\cycle{XXVIII}
\contacts{Via della Sintassi Astratta, 0/1\\65536 Gigatera --- Italia\\+39 0123 456789\\\texttt{http://www.example.com}\\\texttt{inbox@example.com}}
\rights{Tutti i diritti riservati a me stesso e basta.}
%% Campi obbligatori: \title e \author.

%% --- Pacchetti consigliati ---
%% hyperref: Regola le impostazioni della creazione del PDF... pi˘ tante altre cose.
%% tocbibind: Inserisce nell'indice anche la lista delle figure, la bibliografia, ecc.

%% --- Stili di pagina disponibili (comando \pagestyle) ---
%% sfbig (predefinito): Apertura delle parti e dei capitoli col numero grande; titoli delle parti e dei capitoli e intestazioni di pagina in sans serif.
%% big: Come "sfbig", solo serif.
%% plain: Apertura delle parti e dei capitoli tradizionali di LaTeX; intestazioni di pagina come "big".

\begin{document}

\message{ !name(thud.tex) !offset(-3) }



%% Il frontespizio prima di tutto!

\message{ !name(thud.tex) !offset(166) }

\end{document}

--- Istruzioni per l'aggiunta di nuove lingue ---
Per ogni nuova lingua utilizzata aggiungere nel preambolo il seguente spezzone:
    \addto\captionsitalian{%
        \def\abstractname{Sommario}%
        \def\acknowledgementsname{Ringraziamenti}%
        \def\authorcontactsname{Contatti dell'autore}%
        \def\candidatename{Candidato}%
        \def\chairname{Direttore}%
        \def\conclusionsname{Conclusioni}%
        \def\cosupervisorname{Co-relatore}%
        \def\cosupervisorsname{Co-relatori}%
        \def\cyclename{Ciclo}%
        \def\datename{Anno accademico}%
        \def\indexname{Indice analitico}%
        \def\institutecontactsname{Contatti dell'Istituto}%
        \def\introductionname{Introduzione}%
        \def\prefacename{Prefazione}%
        \def\reviewername{Controrelatore}%
        \def\reviewersname{Controrelatori}%
        %% Anno accademico
        \def\shortdatename{A.A.}%
        \def\summaryname{Riassunto}%
        \def\supervisorname{Relatore}%
        \def\supervisorsname{Relatori}%
        \def\thesisname{Tesi di \expandafter\ifcase\csname thud@target\endcsname Laurea\or Laurea Magistrale\or Dottorato\fi}%
        \def\tutorname{Tutor aziendale%
        \def\tutorsname{Tutor aziendali}%
    }
sostituendo a "italian" (nella 1a riga) il nome della lingua e traducendo le varie voci.
